
\documentclass[12pt]{article}

%AMS-TeX packages
\usepackage{amssymb,amsmath,amsthm}
%geometry (sets margin) and other useful packages
\usepackage[margin=1.25in]{geometry}
\usepackage{graphicx,ctable,booktabs}
\usepackage{utopia}

\usepackage{fancyhdr} % Required for custom headers
\usepackage{lastpage} % Required to determine the last page for the footer
\usepackage{extramarks} % Required for headers and footers
\usepackage{graphicx} % Required to insert images
\usepackage{listings} % Required for insertion of code
\usepackage{courier} % Required for the courier font
\usepackage{lipsum} % Used for inserting dummy 'Lorem ipsum' text into the template
\usepackage{verbatim}
\usepackage{alltt}
\usepackage{fancyvrb}
\usepackage{cprotect}

\newcommand{\sep}{-\kern-.6em\raisebox{-.659ex}{*}\ }

\setlength{\parindent}{0em}
\setlength{\parskip}{1.5ex}
	
%
%Fancy-header package to modify header/page numbering
%
\usepackage{fancyhdr}
\pagestyle{fancy}
%\addtolength{\headwidth}{\marginparsep} %these change header-rule width
%\addtolength{\headwidth}{\marginparwidth}
\lhead{}
\chead{}
\rhead{\thepage}
\lfoot{\small\scshape }
\cfoot{}
\rfoot{\footnotesize }
\renewcommand{\headrulewidth}{.3pt}
\renewcommand{\footrulewidth}{.3pt}
\setlength\voffset{-0.25in}
\setlength\textheight{648pt}

\begin{document}
\title{Presentation Notes}
\author{Ana Nora Evans}
\date{\today}
\maketitle
\thispagestyle{empty}

\section{Paper}
These notes are prepared for research group meeting discussing the ASE 2016 paper, \textit{Inferring annotations for device drivers from verification histories} \cite{paper}. 

\section{Program Verifiers}

\textbf{Model checking} is the exhaustive exploration of the state space of a system, typically to see if an error state is reachable. It produces concrete counterexamples.

A program verifier checks if a program satisfies certain properties.

\textbf{SLAM} does verification by model checking. It creates a boolean model of the program and then checks for satisfiability of the formula.

Input: 
\begin{itemize}
\item standard C program 
\item Specification (written in SLIC) given as a finite state machine.
\end{itemize}
Output:
\begin{itemize}
\item Verified which means "program does not violate the given specification" (with a proof).
\item counterexample 
\end{itemize}

\textbf{The Static Driver Verifier}
Static Driver Verifier \cite{SDV} is a verification tool included in the Windows Driver Kit (WDK). It uses SLAM. Among other things, SDV provides a number of class-specific components (for example, API rules and an environment model). API rules are expressed in the SLIC language and describe the proper way to use the driver APIs.

Note, SDV is available for academic use.

\textbf{Corral} \cite{Corral} is an SMT-based verifier that replaced SLAM in SDV.

\textbf{Houdini}

\textbf{Annotations} are candidate predicates. 

\textbf{Invariants} are logical formulas over program variables that are always true. For example: $x>0$. 

\section{Key Insight}
Programs that use the same API probably require similar annotations for verifying contracts of that API.  
By using information from prior verification runs (i.e invariants already checked) of the tool on potentially different sources, candidate invariants will be generated and checked. Previous work, such as SLAM \citep{•}, used only the given specification and the program to be analyzed.

Key challenges:
\begin{itemize}
\item invariants and annotations are formulas over program variables, using local variables that will be out of scope on a different location
\item keep the size of inferred annotations small. The generated set is minimal w.r.t. subset relation and generates all the invariants from previous runs.
\end{itemize}

\bibliographystyle{unsrt}
\bibliography{citations}

\end{document} 